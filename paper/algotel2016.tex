\documentclass[%
% if you want to use pdflatex, uncomment the following line
%pdftex
% if you have difficulties with fonts uncomment the following line
%notimes
]{algotel}
\usepackage[latin1]{inputenc}
\usepackage{ mathrsfs }
\usepackage{mathtools}  
\usepackage{amsmath}
\usepackage{relsize}
\usepackage{lmodern}
\usepackage{slantsc}
\usepackage{amssymb}
\usepackage{tabulary}
\usepackage{booktabs}
\usepackage{multicol}
\usepackage{blindtext}
\usepackage[francais]{babel}
\author{Nicolas Herbaut\addressmark{1}\addressmark{2}\thanks{The work performed for this paper has been partially funded by the FP7 IP T-NOVA European Project (Grant Agreement 619520) and the FUI French National Project DVD2C}{\ }
  \and Daniel N�gru\addressmark{1}}


\title[]{	D�ploiement d'un chaine de service de livraison de contenu multimedia sur une infrastructure d'op�rateur bas�e sur SDN-NFV}

\address{\addressmark{1}Univ. Bordeaux, LaBRI, UMR 5800, F-33400 Talence, France
\addressmark{2}Viotech Communications, Versaille, France

}


\keywords{SDN, NFV, Service Chaining, Content Delivery}

\begin{document}
\maketitle

\begin{abstract}
La consommation de videos "over the top" a chang� la donne pour les acteurs du monde de la livraison de contenu. La chaine de valeur a gliss� progressivement en faveur des fournisseurs de contenus et des r�seaux de livraison de contenu et au d�triment des fournisseurs d'acc�s.
Notre contribution propose un nouveau mod�le de collaboration entre ces diff�rents acteurs par la cr�ation d'une plateforme o� sont d�ploy�es des cha�nes de service de distribution de contenu. Nous �tudions en particuliers les solutions de placement de ces fonctions r�seaux sur un substrat de serveur virtualis�s au sein d'un r�seau respectant l'approche Software Defined Network (SDN) dans lequel sont d�ploy�es des infrastructures de virtualisation de fonction r�seau (NFV).
\end{abstract}

\section{Introduction}
\cite{xu_analysis_2006} is cool.

\section{Model}

minimize

\begin{equation} \label{eq1}
\begin{split}
\max_{
\!\mathsmaller{x \in \mathscr{T} }}{\mathscr{D}(s,x)} 
											 &=\max_{\!\mathsmaller{x \in \mathscr{T}} }{
			\sum_{
			\!\mathsmaller{
			(i,j)\in\mathscr{P}^{S}(s,x)
			}
			}{\mathscr{D}_{\mathscr{M}}(i,j)}}\\
											&=\max_{x \in \mathscr{T} }{
			\sum_{
			\!\mathsmaller{
			(i,j)\in\mathscr{P}^{S}(s,x)
			}
			}}{\quad\sum_{
			\!\mathsmaller{
			{(a,b)\in\mathscr{P}^{S}_{\mathscr{M}}(i,j)}
			}
			}{d_{a,b}\times y_{a,b}^{i,j}}}\\
\end{split}	
\end{equation}

\begin{equation}
		\sum_{u\in N} x_u^{i}=1, \forall i \in N^{S}
\end{equation}


\begin{equation}
	\sum_{i \in N^{S} } x_{j}^{i} \times c_{i}^{S} \leq c_{u}, \forall u \in N		
\end{equation}

\begin{equation}
	\sum_{(i,j)\in E^{S}}{y_{u,v}^{i,j}\times b_{i,j}^{S}} \leq b_{u,v}, \forall (u,v) \in E
\end{equation}

\begin{equation}
	y_{u,v}^{i,j}\times d_{u,v} \leq d_{i,j}^{S}, \forall (u,v) \in E, \forall (i,j) \in E^{S}
\end{equation}


\begin{equation}
\sum_{v \in N}{y_{u,v}^{i,j}-y_{v,u}^{i,j}} = x_{u}^{i}-x_{u}^{j}, \forall (i,j) \in E^{S}, \forall u \in N
\end{equation}


\begin{equation}
y_{u,v}^{i,j}+y_{v,u}^{i,j} \leq 1, \forall (u,v) \in E, \forall (i,j) \in E^{S}
		s
\end{equation}

 \begin{multicols}{2}[\textbf{Example for a two column text}]
    \blindtext
  \end{multicols}

  \begin{multicols}{3}[\textbf{Example for a three column text}]
    \blindtext
  \end{multicols}
  

\begin{table}[htbp]\caption{Notations}
\begin{center}% used the environment to augment the vertical space
% between the caption and the table
\begin{tabular}{r c p{10cm} }
\toprule{}


$y_{u,v}^{i,j}$ & $=$ & \(
	\left \{
		\begin{array}{rl}
			1,  & \text{if link $(i,j) \in E^{S}$ is mapped on substrate link $(a,b)\in E$} \\
			0,  & \text{otherwise} 
		\end{array}
	\right.\)\\

$d_{a,b}$ & $\triangleq$ & the delay for edge $(a,b)\in E$\\
$\mathscr{D}_{\mathscr{M}}(i,j)$ & $\triangleq$ & the total substrate delay for mapping $\mathscr{M}$ for $(i,j) \in E^{S}$ \\
$\mathscr{T}$ & $\triangleq$ & the set of terminal nodes of the chain.\\
$s$ & $\triangleq$ & the starting node of the chain.\\
$i \xrightarrow{\mathscr{M}} u$ & $\triangleq$ & the network function $i\in N^{S}$ is mapped to the substrate node $u\in N$ for mapping $\mathscr{M}$ \\

$\mathscr{P}^{S}(i,j)$ & $\triangleq$ & path from $i$ to $j$ is the set of edges $e\in E^{S}$ that links $i \in N^{S}$ to $j \in N^{S}$\\

$\mathscr{P}^{S}_{\mathscr{M}}(i,j)$ & $\triangleq$ & mapped path from i to j is the set of edges $e\in E$ that links $u$ to $v$ with $i \xrightarrow{\mathscr{M}} u$ and $j \xrightarrow{\mathscr{M}} v$\\


\bottomrule
\end{tabular}
\end{center}
\label{tab:TableOfNotationForMyResearch}
\end{table}


\nocite{}
\bibliographystyle{alpha}
\bibliography{vnf}



\end{document}

\documentclass[%
% if you want to use pdflatex, uncomment the following line
%pdftex
% if you have difficulties with fonts uncomment the following line
%notimes
]{algotel}
\usepackage[latin1]{inputenc}
\usepackage{ mathrsfs }
\usepackage[francais]{babel}
\author{Nicolas Herbaut\addressmark{1}\addressmark{2}\thanks{The work performed for this paper has been partially funded by the FP7 IP T-NOVA European Project (Grant Agreement 619520) and the FUI French National Project DVD2C}{\ }
  \and Daniel N�gru\addressmark{1}}


\title[]{	D�ploiement d'un chaine de service de livraison de contenu multimedia sur une infrastructure d'op�rateur bas�e sur SDN-NFV}

\address{\addressmark{1}Univ. Bordeaux, LaBRI, UMR 5800, F-33400 Talence, France
\addressmark{2}Viotech Communications, Versaille, France

}


\keywords{SDN, NFV, Service Chaining, Content Delivery}

\begin{document}
\maketitle

\begin{abstract}
La consommation de videos "over the top" a chang� la donne pour les acteurs du monde de la livraison de contenu. La chaine de valeur a gliss� progressivement en faveur des fournisseurs de contenus et des r�seaux de livraison de contenu et au d�triment des fournisseurs d'acc�s.
Notre contribution propose un nouveau mod�le de collaboration entre ces diff�rents acteurs par la cr�ation d'une plateforme o� sont d�ploy�es des cha�nes de service de distribution de contenu. Nous �tudions en particuliers les solutions de placement de ces fonctions r�seaux sur un substrat de serveur virtualis�s au sein d'un r�seau SDN.
\end{abstract}

\section{Introduction}
\cite{xu_analysis_2006} is cool.

\section{Model}

\[
\max_{x \in \mathscr{T} }{\mathscr{D}_{s,x}}=\sum_{(a,b)\in\mathscr{D}^{\mathscr{M}}}
\]



\nocite{}
\bibliographystyle{alpha}
\bibliography{vnf}



\end{document}
